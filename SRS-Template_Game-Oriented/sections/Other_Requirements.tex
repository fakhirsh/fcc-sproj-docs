
\section{Other Requirements}

% Content goes here
\subsection{User Interface Mockups}

Here are some user interface mockups for the system.

\begin{figure}[htbp]
    \centering
    \includegraphics[width=0.8\textwidth]{images/mockup}
    \caption{Mockup 1: Login Screen}
    \label{fig:mockup1}
\end{figure}

\begin{figure}[htbp]
    \centering
    \includegraphics[width=0.8\textwidth]{images/mockup}
    \caption{Mockup 2: Dashboard}
    \label{fig:mockup2}
\end{figure}

\subsection{Performance Requirements}

The system should be able to handle a minimum of 1000 concurrent users without any significant degradation in performance. Response time for common operations should be less than 1 second.

\subsection{Security Requirements}

The system should implement secure authentication and authorization mechanisms to protect user data. Passwords should be stored securely using industry-standard encryption algorithms.

\subsection{Availability Requirements}

The system should have a minimum uptime of 99.9%. Scheduled maintenance windows should be communicated to users in advance.

\subsection{Documentation Requirements}

The system should have comprehensive documentation that includes installation instructions, user guides, and API documentation.

\subsection{Legal Requirements}

The system should comply with all applicable laws and regulations, including data protection and privacy laws.

\subsection{Training Requirements}

The system should be intuitive and easy to use, requiring minimal training for end users. However, a training program should be provided to administrators and support staff to ensure they can effectively manage and troubleshoot the system.

\subsection{Localization Requirements}

The system should support multiple languages and provide localization options for date formats, currency symbols, and other region-specific settings.

\subsection{Usability Requirements}

The system should have a clean and intuitive user interface, with consistent navigation and clear error messages. It should be accessible to users with disabilities, following WCAG 2.0 guidelines.

\subsection{Compatibility Requirements}

The system should be compatible with modern web browsers (Chrome, Firefox, Safari, Edge) and mobile devices (iOS, Android).

\subsection{Scalability Requirements}

The system should be designed to scale horizontally, allowing for easy addition of new servers to handle increased load. It should also support vertical scaling by utilizing resources efficiently.

\subsection{Maintainability Requirements}

The system should be modular and well-documented, allowing for easy maintenance and future enhancements. Code should follow best practices and be thoroughly tested.

\subsection{Support Requirements}

The system should have a dedicated support team to address user inquiries and provide timely assistance. Support should be available during business hours and include a ticketing system for issue tracking.

\subsection{Backup and Recovery Requirements}

The system should have regular backups of data to prevent data loss in case of hardware failure or other disasters. A recovery plan should be in place to restore the system to a functional state in the event of a failure.

\subsection{Performance Monitoring Requirements}

The system should have performance monitoring tools in place to track system metrics, identify bottlenecks, and optimize performance as needed.

\subsection{Third-Party Integrations}

The system should be able to integrate with external systems, such as payment gateways, email services, and CRM systems, as required by the business.

\subsection{Testing Requirements}

The system should undergo thorough testing, including unit tests, integration tests, and user acceptance testing, to ensure its functionality and reliability.

\subsection{Deployment Requirements}

The system should have a well-defined deployment process, including version control, continuous integration, and automated deployment pipelines.

\subsection{Change Management Requirements}

The system should have a change management process in place to handle updates, bug fixes, and new feature releases. Changes should be properly documented and tested before being deployed to production.